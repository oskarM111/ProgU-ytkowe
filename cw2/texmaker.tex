
\documentclass[12pt, letterpaper, titlepage]{article}
\usepackage[left=3.5cm, right=2.5cm, top=2.5cm, bottom=2.5cm]{geometry}
\usepackage[MeX]{polski}
\usepackage[utf8]{inputenc}
\usepackage{graphicx}
\usepackage{enumerate}
\usepackage{amsmath} %pakiet matematyczny
\usepackage{amssymb} %pakiet dodatkowych symboli
\title{METODY NAUKOWE W METODOLOGII NAUK O ZARZĄDZANIU}
\author{Oskar Maik}
\date{15.10.2022}
\begin{document}
\maketitle
\begin{enumerate}
\item punkt1
\item punkt2
\end{enumerate}

\begin{enumerate}[A]
\item punkt1
\item punkt2
\end{enumerate}

\section{Abstrakt
Artykuł przedstawia charakterystykę dwóch podstawowych metod naukowych wykorzystywanych w naukach empirycznych. Metody indukcyjna i hipotetyczno-dedukcyjna, jako metody nauk empirycznych, są w nim przedstawiane z perspektywy ogólnej metodologii nauk z uwzględnieniem specyfiki nauk o zarządzaniu. Pracę dopełniają rozważania określające wytyczne wykorzystania metod naukowych w obszarze tej nauki.}
 \subsection{Abstrakt
Artykuł przedstawia charakterystykę dwóch podstawowych metod naukowych wykorzystywanych w naukach empirycznych. Metody indukcyjna i hipotetyczno-dedukcyjna, jako metody nauk empirycznych, są w nim przedstawiane z perspektywy ogólnej metodologii nauk z uwzględnieniem specyfiki nauk o zarządzaniu. Pracę dopełniają rozważania określające wytyczne wykorzystania metod naukowych w obszarze tej nauki.}
Tekst dotyczący pierwszej podsekcji
\subsubsection{Abstrakt
Artykuł przedstawia charakterystykę dwóch podstawowych metod naukowych wykorzystywanych w naukach empirycznych. Metody indukcyjna i hipotetyczno-dedukcyjna, jako metody nauk empirycznych, są w nim przedstawiane z perspektywy ogólnej metodologii nauk z uwzględnieniem specyfiki nauk o zarządzaniu. Pracę dopełniają rozważania określające wytyczne wykorzystania metod naukowych w obszarze tej nauki.}
Tekst dotyczący pod-podsekcji
\end{document}
